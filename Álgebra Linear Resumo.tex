% Options for packages loaded elsewhere
\PassOptionsToPackage{unicode}{hyperref}
\PassOptionsToPackage{hyphens}{url}
%
\documentclass[
]{article}
\usepackage{amsmath,amssymb}
\usepackage{lmodern}
\usepackage{iftex}
\ifPDFTeX
  \usepackage[T1]{fontenc}
  \usepackage[utf8]{inputenc}
  \usepackage{textcomp} % provide euro and other symbols
\else % if luatex or xetex
  \usepackage{unicode-math}
  \defaultfontfeatures{Scale=MatchLowercase}
  \defaultfontfeatures[\rmfamily]{Ligatures=TeX,Scale=1}
\fi
% Use upquote if available, for straight quotes in verbatim environments
\IfFileExists{upquote.sty}{\usepackage{upquote}}{}
\IfFileExists{microtype.sty}{% use microtype if available
  \usepackage[]{microtype}
  \UseMicrotypeSet[protrusion]{basicmath} % disable protrusion for tt fonts
}{}
\makeatletter
\@ifundefined{KOMAClassName}{% if non-KOMA class
  \IfFileExists{parskip.sty}{%
    \usepackage{parskip}
  }{% else
    \setlength{\parindent}{0pt}
    \setlength{\parskip}{6pt plus 2pt minus 1pt}}
}{% if KOMA class
  \KOMAoptions{parskip=half}}
\makeatother
\usepackage{xcolor}
\IfFileExists{xurl.sty}{\usepackage{xurl}}{} % add URL line breaks if available
\IfFileExists{bookmark.sty}{\usepackage{bookmark}}{\usepackage{hyperref}}
\hypersetup{
  hidelinks,
  pdfcreator={LaTeX via pandoc}}
\urlstyle{same} % disable monospaced font for URLs
\setlength{\emergencystretch}{3em} % prevent overfull lines
\providecommand{\tightlist}{%
  \setlength{\itemsep}{0pt}\setlength{\parskip}{0pt}}
\setcounter{secnumdepth}{-\maxdimen} % remove section numbering
\ifLuaTeX
  \usepackage{selnolig}  % disable illegal ligatures
\fi

\author{}
\date{}

\begin{document}

\hypertarget{uxe1lgebra-linear-principais-ideias}{%
\section{Álgebra Linear: Principais
Ideias}\label{uxe1lgebra-linear-principais-ideias}}

\emph{Aviso: este material está em construção. Ele é escrito por
\href{https://adairneto.github.io/}{Adair Antonio da Silva Neto}, aluno
de bacharelado em Matemática, para estudo próprio, mas na expectativa de
que ajude outras pessoas em seu aprendizado.}

\hypertarget{estruturas-alguxe9bricas}{%
\subsection{Estruturas Algébricas}\label{estruturas-alguxe9bricas}}

\hypertarget{conceitos-e-resultados-principais-1}{%
\subsubsection{Conceitos e Resultados
Principais}\label{conceitos-e-resultados-principais-1}}

Corpo, Corpo Ordenado, Números Reais, Números Complexos, Métrica.

\textbf{O que é um corpo (ou field)?}

Antes de definir o que é um corpo, vamos considerar um exemplo de um
corpo.

Seja \(\mathbb{F}\) o conjunto dos números reais ou complexos. Sabemos
que:

\begin{enumerate}
\def\labelenumi{\arabic{enumi}.}
\item
  A adição é comutativa: \(x+y = y+x\), \(\forall x, y \in \mathbb{F}\).
\item
  A adição é associativa \(x + (y+z) = (x+y) + z\),
  \(\forall x, y, z \in \mathbb{F}\).
\item
  Existe um único elemento \(0 \in \mathbb{F}\) tal que \(x+0 = x\),
  para todo \(x\) em \(\mathbb{F}\) (elemento neutro).
\item
  Para cada \(x \in \mathbb{F}\), existe um único elemento
  correspondente \((-x) \in \mathbb{F}\) tal que \(x + (-x) = 0\)
  (elemento simétrico).
\item
  A multiplicação é comutativa \(xy = yx\),
  \(\forall x,y \in \mathbb{F}\).
\item
  A multiplicação é associativa \(x(yz) = (xy)z\),
  \(\forall x, y, z \in \mathbb{F}\).
\item
  Existe um único elemento \(1 \in \mathbb{F}\) tal que \(x1 = x\),
  \(\forall x \in \mathbb{F}\) (elemento neutro).
\item
  Para cada elemento \(x \in \mathbb{F}\) diferente de zero, existe um
  único elemento correspondente \(x^{-1} = (1/x)\) tal que
  \(xx^{-1} = 1\) (inverso multiplicativo).
\item
  Distributividade: \(x(y+z) = xy + xz\),
  \(\forall x,y,z \in \mathbb{F}\).
\end{enumerate}

A motivação por trás da definição de corpo é generalizar essas
propriedades para outros conjuntos e outras operações. Ou seja, queremos
trabalhar com conjuntos, munidos de duas operações que "funcionam" como
a adição e a multiplicação dos reais.

Assim, dizemos que um corpo \(\mathbb{F}\) é um conjunto não vazio
dotado de duas operações \(+\) e \(\times\) satisfazendo as propriedades
1-9.

Há ainda uma restrição que precisamos fazer. Queremos que toda operação
de adição ou de multiplicação nos leve a um elemento dentro do corpo.
Isto é, queremos que o corpo seja fechado quanto à adição e
multiplicação. Formalmente,

(F1) Para todos \(x,y \in \mathbb{F}\), \(x+y \in \mathbb{F}\).\\
(F2) para todos \(x,y \in \mathbb{F}\), \(x \times y \in \mathbb{F}\).

Esses são chamados axiomas de fechamento. As propriedades 1-4 são
chamadas axiomas da operação de adição e as propriedades 5-9, axiomas da
operação de multiplicação.

\textbf{O que é um corpo ordenado?}

Um corpo \(\mathbb{F}\) munido de uma relação de ordem \(<\) é chamado
de corpo ordenado se ele satisfaz os seguintes axiomas:

(O1) Princípio da Comparação: Queremos que apenas um dos seguintes casos
aconteça: ou \(x<y\), ou \(y<x\) ou \(x=y\).\\
(O2) Transitividade: \(x < y \land y < z \implies x < z\).\\
(O3) Consistência da adição: \(y < z \implies x+y < x + z\).\\
(O4) Consistência da multiplicação:
\((0 < x \land 0 < y) \implies 0 < x \times y\).

Onde \(x, y, z \in \mathbb{F}\).

\textbf{Números complexos}

Antes de tudo, como podemos definir o conjunto dos números complexos
\(\C\)?

\[\C = \{ a+bi : a, b \in \R \}\]

\textbf{Como somar números complexos?}

Seja \(z = a + bi\) e \(w = c+di \in \C\). Definimos a sua soma como

\[z+w = (a+bi) + (c+di) = (a+c)+(b+d)i\]

\textbf{Como multiplicar números complexos?}

E definimos seu produto como

\[z \cdot w = (a+bi) \cdot (c+di) = (ac-bd) + (bc+ad)i\]

Note que
\(i^2 = i \cdot i = (0+1i) \cdot (0+1i) = (0-1) + (0+0)i = -1\).

O número \(i\) é chamado imaginário puro.

\textbf{Representação geométrica de \(\C\)}

Dado \(\R^2\) o plano cartesiano usual, identificamos o número complexo
\(z = a+bi\) com o ponto \((a, b) \in \R^2\).

Com essa representação geométrica em mente, vem do Teorema de Pitágoras
que o \textbf{valor absoluto} (ou \textbf{módulo}) de um número complexo
\(z\) é

\[|z| = \sqrt{a^2 + b^2}\]

Em coordenadas polares, temos \(a = r \cos \theta\) e
\(b = r \sin \theta\).

\textbf{Complexo conjugado}

Geometricamente, o conjugado complexo de \(z\) é a reflexão do ponto
\(z\) em relação ao eixo real \(Ox\). Simbolicamente,

\[\bar{z} := a-bi\]

\textbf{O que é uma métrica?}

Como podemos generalizar a noção que temos de distância?

Intuitivamente uma distância exige um ponto inicial e um ponto final,
resultando, a partir desses pontos, um número real.

Sabemos que toda distância é simétrica (isto é, têm o mesmo valor
independentemente do sentido) e sempre positiva. Além disso, queremos
que, dados três pontos \(x, y, z\), a distância de \(x\) até \(z\) seja
menor ou igual a distância de \(x\) até \(y\) mais a distância de \(y\)
até \(z\).

Assim, dado um conjunto não vazio qualquer \(\mathbb{X}\), podemos
definir uma métrica (ou distância) como sendo uma função
\(d(\cdot , \cdot): \mathbb{X} \times \mathbb{X} \to \R\) satisfazendo:

\begin{enumerate}
\def\labelenumi{\arabic{enumi}.}
\item
  Simetria: \(d(x,y) = d(y,x)\).
\item
  Positividade: \(d(x,y) \geq 0\), sendo que \(d(x,y) = 0 \iff x = y\).
\item
  Desigualdade Triangular: \(d(x,z) \leq d(x,y) + d(y,z)\).
\end{enumerate}

Onde \(x,y,z\) são elementos quaisquer do conjunto \(\mathbb{X}\).

Vamos utilizar a notação \((\mathbb{X},d)\) para indicar que o conjunto
\(\mathbb{X}\) possui a métrica \(d(\cdot , \cdot)\). Dizemos que
\((\mathbb{X},d)\) é um \textbf{espaço métrico}.

\textbf{Função Traço (Trace)}

É a soma dos elementos da diagonal principal de uma matriz quadrada.
Isto é,

\[tr A = \sum_{i=1}^n {a_{ii}}\]

onde \(A = (a_{ij})_{i,j} \in \mathbb{M}_n(\mathbb{K})\).

\textbf{Posto de uma Matriz (Matrix Rank)}

É o número de linhas não nulas de uma matriz em sua forma escalonada.

\textbf{Matrizes adjuntas}

É a transposta da matriz (quadrada!) dos cofatores.

\hypertarget{material-complementar-1}{%
\subsubsection{Material Complementar}\label{material-complementar-1}}

O vídeo
\href{https://www.youtube.com/watch?v=fNk_zzaMoSs\&list=PLZHQObOWTQDPD3MizzM2xVFitgF8hE_ab\&index=1}{Vetores,
o que são eles afinal?} é uma ótima introdução ao que vamos estudar
nesta disciplina.

\hypertarget{espauxe7o-vetorial}{%
\subsection{Espaço Vetorial}\label{espauxe7o-vetorial}}

\hypertarget{conceitos-e-resultados-principais-2}{%
\subsubsection{Conceitos e Resultados
Principais}\label{conceitos-e-resultados-principais-2}}

Vamos introduzir aqui o objeto matemático que será o centro de nosso
estudo em Álgebra Linear. A ideia é trabalhar com um sistema algébrico
que generalize a noção de combinação linear num dado conjunto.

\textbf{O que é um espaço vetorial?}

Um \textbf{espaço vetorial} consiste de:

\begin{enumerate}
\def\labelenumi{\arabic{enumi}.}
\item
  Um conjunto \(\mathbb{F}\) de \textbf{escalares};
\item
  Um conjunto \(\mathbb{V}\) de objetos chamados \textbf{vetores};
\item
  Uma regra (i.e. operação) chamada \textbf{adição de vetores}, que
  associa a cada par de vetores \(\alpha , \beta \in \mathbb{V}\) um
  vetor \(\alpha + \beta \in \mathbb{V}\) tal que as seguintes
  propriedades são satisfeitas:

  \begin{enumerate}
  \def\labelenumii{\arabic{enumii}.}
  \item
    Comutatividade: \(\alpha + \beta = \beta + \alpha\);
  \item
    Associatividade:
    \(\alpha + (\beta + \gamma) = (\alpha + \beta) + \gamma\);
  \item
    Elemento neutro: Existe um único vetor \(0 \in \mathbb{V}\) tal que
    \(\alpha + 0 = \alpha, \forall \alpha \in \mathbb{V}\);
  \item
    Elemento simétrico: Para cada vetor \(\alpha\), existe um único
    vetor \(-\alpha\) tal que \(\alpha + (- \alpha) = 0\);
  \end{enumerate}
\item
  Uma regra chamada \textbf{multiplicação por escalar} que associa o par
  \(c \in \mathbb{F}\) e \(\alpha \in \mathbb{V}\) ao vetor
  \(c \alpha \in \mathbb{V}\), satisfazendo:

  \begin{enumerate}
  \def\labelenumii{\arabic{enumii}.}
  \item
    Elemento identidade:
    \(1 \alpha = \alpha, \forall \alpha \in \mathbb{V}\);
  \item
    Associatividade: \((c_1 c_2)\alpha = c_1(c_2 \alpha)\);
  \item
    Distributividade para adição de vetores:
    \(c(\alpha + \beta) = c\alpha + c\beta\);
  \item
    Distributividade para a multiplicação por escalar:
    \((c_1 + c_2) \alpha = c_1 \alpha + c_2 \alpha\).
  \end{enumerate}
\end{enumerate}

Os objetos que chamamos de vetores num espaço vetorial podem não ser os
vetores aos quais estamos acostumados do Ensino Médio. Podem ser
matrizes, funções, polinômios etc.

\textbf{O que é combinação linear?}

Falamos em combinação linear anteriormente remetendo à noção intuitiva
que temos de Geometria Analítica. Formalmente, dizemos que um vetor
\(\beta\) é dito \textbf{combinação linear} dos vetores
\((\alpha_1, \alpha_2, \ldots, \alpha_n)\) se existirem escalares
\(c_1, \ldots, c_n\) tais que \(\beta\) pode ser escrito como

\[\beta = c_1 \alpha_1 + \ldots + c_n \alpha_n\]

Ou, em notação mais compacta,

\[\beta = \sum_{i=1}^{n} c_i \alpha_i\]

\textbf{O que são subespaços?}

Considere \(V\) um espaço vetorial sobre o corpo \(F\). Um subconjunto
\(W \subset V\) é dito um \textbf{subespaço} de \(V\) se \(W\) for um
espaço vetorial sobre \(F\) com as operações de soma de vetores e
multiplicação por escalar.

\textbf{O que é uma matriz hermitiana (ou matriz auto-adjunta)?}

É uma matriz quadrada \(A\), sobre o corpo \(\C\), tal que
\(A = \overline{A^T}\), i.e. cada \(a_{ij}= \overline{a_{ji}}\). É o
análogo complexo da matriz simétrica.

Para uma matrix \(2 \times 2\), uma matriz é hermitiana sse. for da
forma

\begin{bmatrix}
	z & x+iy \\
	x-iy & w
\end{bmatrix}

onde \(x, y, z, w \in \R\).

\textbf{O que significa subespaço gerado (subspace spanned)?}

Dado um conjunto de vetores \(S\) em um espaço vetorial \(V\), dizemos
que o \textbf{subespaço gerado} por \(S\) é a interseção de todos os
subespaços de \(V\) que contêm \(S\).

Caso \(S\) seja um conjunto não nulo, então o subespaço gerado por ele
(vamos chamá-lo de \(W\)) é o conjunto de todas as combinações lineares
de vetores em \(S\). Isto é, a partir dos vetores de \(S\) eu consigo
escrever qualquer vetor em \(W\).

\textbf{O que são base e dimensão?}

Uma \textbf{base} para o espaço vetorial \(V\) é um conjunto de vetores
linearmente independentes que geram o espaço (\emph{span the space})
\(V\). Vamos dizer que o espaço \(V\) tem \textbf{dimensão finita} se
ele possui uma base finita.

Por isso, os vetores \(e_1 = (1,0,0,\ldots,0)\),
\(e_2 = (0,1,0,\ldots,0)\), \(\ldots\), \(e_n = (0,0,0,\ldots,1)\)
formam uma base do espaço \(\R^n\) e são chamados de \textbf{base
canônica} de \(\R^n\).

Observe que a definição de base implica que se \(V\) é um espaço
vetorial gerado por um conjunto de \(m\) vetores, então qualquer
conjunto de vetores linearmente independentes em \(V\) é finito e contém
no máximo \(m\) elementos. Consequentemente, quaisquer duas bases de
\(V\) têm o mesmo número \(m\) de elementos.

Com isso em mente, vamos definir \textbf{dimensão} de um espaço vetorial
finito \(V\) como sendo o número de elementos da base de \(V\) e
denotamos \(\text{dim}(V)\).

Essas definições têm como consequência dois fatos importantes:

\begin{enumerate}
\def\labelenumi{\arabic{enumi}.}
\item
  Qualquer subconjunto de \(V\) com mais de \(\text{dim}(V)\) elementos
  é linearmente dependente.
\item
  Nenhum subconjunto de \(V\) que tem menos que \(\text{dim}(V)\)
  elementos pode gerar \(V\).
\end{enumerate}

Caso \(W_1\) e \(W_2\) sejam dois subespaços finitos de \(V\), então
\(W_1 + W_2\) tem dimensão finita e

\[\text{dim}(W_1) + \text{dim}(W_2) = \text{dim}(W_1 \cap W_2) + \text{dim}(W_1 + W_2)\]

\textbf{O que são coordenadas?}

As coordenadas de um vetor \(v \in V\), relativas à base \(\beta\), são
os coeficientes que permitem expressar \(v\) como uma combinação linear
dos vetores de \(\beta\). As coordenadas mais naturais são aquelas que
utilizam a base canônica do corpo \(\mathbb{F}\) no qual estamos
trabalhando. Ou seja,

\[v = (x_1, \ldots, x_n) = \sum x_i e_i\]

onde \(e_i\) é o i-ésimo elemento da base canônica.

Para trabalhar com mudança de coordenadas, precisamos antes entender o
que é uma \textbf{base ordenada} de um espaço de dimensão finita \(V\).
Dizemos que \(\beta\) é uma base ordenada de \(V\) se \(\beta\) é uma
sequência finita de vetores que é linearmente independente e gera \(V\).

Note que uma base é um conjunto, um objeto no qual a ordem não faz
diferença. Porém, uma base ordenada é uma sequência, o que nos permite
distinguir quem é seu i-ésimo elemento. Assim, se \(a_1, \ldots, a_n\) é
uma base ordenada de \(V\), então \(\{ a_1, \ldots, a_n \}\) é uma base
de \(V\).

Vamos denotar \([ v ]_\beta\) as coordenadas do vetor \(v\) em relação à
base ordenada \(\beta\). Além disso, vamos fazer um abuso de notação
utilizando \(\beta = \{ a_1, \ldots, a_n \}\) para denotar uma base
ordenada.

\textbf{Como fazer matriz de mudança de base?}

Vamos tomar \(\beta = \{ a_1, \ldots, a_n \}\) e
\(\gamma = \{ b_1, \ldots, b_n \}\) duas bases ordenadas de um espaço
n-dimensional \(V\). Note que podemos escrever cada vetor da base
\(\gamma\) como combinação linear dos vetores de \(\beta\) da seguinte
forma:

\[b_1 = \alpha_{11}\cdot b_1 + \alpha_{21}\cdot b_2 + \ldots + \alpha_{n1}\cdot b_n\\
b_2 = \alpha_{12}\cdot b_1 + \alpha_{22}\cdot b_2 + \ldots + \alpha_{n2}\cdot b_n\\
\vdots \\
b_n = \alpha_{1n}\cdot b_1 + \alpha_{2n}\cdot b_2 + \ldots + \alpha_{nn}\cdot b_n\\\]

onde cada \(\alpha_{ij}\) é um escalar no corpo \(\mathbb{F}\). Assim,
para cada \(i \in \{1, 2, \ldots, n \}\), o vetor das coordenadas de
\(b_i\) na base \(\gamma\) é dado por

\[[c_i]_\beta =
\begin{bmatrix} 
\alpha_{1i} \\
\alpha_{2i} \\
\vdots \\
\alpha_{ni} \\
\end{bmatrix}\]

Dessa maneira conseguimos obter as coordenadas de cada vetor da base
\(\gamma\) em relação à base \(\beta\). Com isso, montamos a
\textbf{matriz de mudança de base} de \(\beta\) para \(\gamma\):

\[P_{\beta,\gamma} =
\begin{bmatrix} 
\alpha_{11} && \ldots && \alpha_{1n} \\
\alpha_{21} && \ldots && \alpha_{2n} \\
\vdots && \ddots && \vdots \\
\alpha_{n1} && \ldots && \alpha_{nn} \\
\end{bmatrix}\]

Note que cada coluna é formada pelas coordenadas de \(b_1, \ldots, b_n\)
em relação à base \(\beta\).

\emph{Rascunho:}

Com essa matriz... podemos escrever...

Existe uma única matriz invertível \(P\), chamada \textbf{matriz de
mudança de base} tal que

\[[v]_\beta = P[v]_{\beta'} \\
[v]_{\beta'} = P^{-1}[v]_\beta\]

Onde cada coluna \(P_i\) é dada pelo vetor de coordenadas
\([v_i']_\beta\) para \(i = 1, \ldots, n\).

Podemos denotar a matriz que leva os vetores na base \(\beta'\) para a
base \(\beta\) como \(P_\beta^{\beta'}\) ou \(P_{\beta',\beta}\).

\hypertarget{exercuxedcios-resolvidos-1}{%
\subsubsection{Exercícios Resolvidos}\label{exercuxedcios-resolvidos-1}}

Como verificar se \(S_A\) é subespaço vetorial de \(S\)?

Basta verificar se \(c \alpha + d \beta \in S_A\), onde
\(c,d \in \mathbb{F}\) e \(\alpha, \beta \in \mathbb{V}\).

\hypertarget{cuxf3digos-no-matlab}{%
\subsubsection{Códigos no MatLab}\label{cuxf3digos-no-matlab}}

\hypertarget{material-complementar-2}{%
\subsubsection{Material Complementar}\label{material-complementar-2}}

Soma Direta e Interseção, Dimensão, Mudança de Base

\href{https://www.youtube.com/watch?v=k7RM-ot2NWY\&list=PLZHQObOWTQDPD3MizzM2xVFitgF8hE_ab\&index=2}{Combinações
lineares, subespaços gerados, e bases}

\hypertarget{transformauxe7uxf5es-lineares}{%
\subsection{Transformações
Lineares}\label{transformauxe7uxf5es-lineares}}

\hypertarget{conceitos-e-resultados-principais-3}{%
\subsubsection{Conceitos e Resultados
Principais}\label{conceitos-e-resultados-principais-3}}

\textbf{O que é uma transformação linear?}

É um mapa que leva um vetor do espaço vetorial \(V\) em um elemento de
do espaço \(W\). Ou seja, uma \textbf{transformação linear} de \(V\) em
\(W\) é uma função

\[T(cv+w) = cT(v)+T(w)\]

onde \(v, w \in V\) e \(c\) é um escalar em \(\mathbb{F}\).

\textbf{Exemplo:} Dado o corpo \(\R\) e \(V\) o espaço formado pelas
funções de \(\R \to \R\) contínuas, podemos definir uma transformação
linear \(T(f(x)) = \int_0^x f(t)dt\). Lembre-se que a linearidade da
integração é uma de suas principais propriedades.

É importante notar que para toda transformação linear \(T\) é verdade
que \(T(0) = 0\). Isto é, \(T\) sempre passa pela origem.

Outro fato importante, que vem direto da definição, é que transformações
lineares preservam combinações lineares. Ou seja,

\[T(c_1v_1+\ldots+c_nv_n)=c_1T(v_1)+\ldots+c_nT(v_n)\]

Muitas vezes encontramos o problema inverso de ter que encontrar a
transformação \(T\) a partir de suas aplicações em vetores
\(e_1, \ldots, e_n\). Um resultado útil nesses casos é o seguinte:

Dada uma base ordenada para \(V\) então existe uma única transformação
linear de \(V\) em \(W\) que leva cada vetor da base ordenada de \(V\)
em um vetor de \(W\).

\textbf{O que são Núcleo e Imagem?}

Considere uma transformação linear \(T:V\to W\). Temos que a
\textbf{imagem} de \(T\) é um subespaço de \(W\) chamado de imagem de
\(T\) e denotamos \(Im(T)\). Pois bem, quais são os elementos de
\(Im(T)\)? São todos os vetores em \(W\) tais que algum vetor
\(v \in V\) é levado até ele pela transformação \(T\). Isto é,

\[Im(T) = \{ w \in W : T(v) = w, v \in V \}\]

Outro subespaço bastante útil associado à \(T\) é aquele que contém
todos os vetores \(v \in V\) que levam ao vetor nulo. Ou seja, todos os
\(v\) tais que \(T(v) = 0\). Esse subespaço é chamado \textbf{núcleo}
(ou \textbf{kernel}) de \(T\) e é definido por

\[Ker(T) = \{ v \in V : T(v) = 0 \}\]

\textbf{O que são Posto e Nulidade?}

Dado um espaço de dimensão finita \(V\), dizemos que o \textbf{posto}
(ou \textbf{rank}) de uma transformação \(T\) é a dimensão da imagem de
\(T\). Também dizemos que a \textbf{nulidade} (ou \textbf{nullity}) de
\(T\) é a dimensão do núcleo de \(T\).

Essas duas definições são relacionadas pelo seguinte

\emph{Teorema do Núcleo e Imagem.} Se \(V\) e \(W\) são dois espaços
vetoriais sobre o corpo \(\mathbb{F}\) e \(T\) é uma transformação
linear de \(V\) em \(W\). Supondo que \(V\) tem dimensão finita, então

\[\text{posto}(T) + \text{nulidade}(T) = \text{dim}(V)\]

Em outras palavras,

\[\text{dim(V)} = \text{dim}(Ker(T)) + \text{dim}(Im(T))\]

Intuitivamente, o \textbf{posto} de uma matriz é o número de linhas ou
colunas linearmente independentes da matriz. É um resultado importante
da Álgebra Linear que o posto das linhas de uma matriz é igual ao posto
de suas colunas.

\textbf{A Álgebra das Transformações Lineares}

É importante e bonito notar que o conjunto das transformações lineares
herda uma estrutura natural do espaço vetorial. Isso ficará claro com a
generalização que faremos agora.

Considerando novamente \(V, W\) espaços vetoriais sobre um corpo
\(\mathbb{F}\) e \(T\) e \(U\) transformações lineares. Temos que

\[(T+U)(v) = T(v)+U(v) \\
(cT)(v) = cT(v)\]

são ambas transformações lineares.

Com isso, vem que o conjunto de todas as transformações lineares de
\(V\) em \(W\), com as operações de adição e multiplicação por escalar
conforme definidos acima, são um espaço vetorial sobre \(\mathbb{F}\),
que denotaremos \(L(V,W)\).

Um importante resultado sobre esse espaço é que se \(V\) tem dimensão
\(n\) e \(W\) tem dimensão \(m\), então a dimensão de \(L(V,W) = mn\).

Além disso, se definirmos \(T:V \to W\) e \(U : W \to Z\), ambas sobre o
mesmo corpo \(\mathbb{F}\), então a composição \(U(T(v))\) é uma
transformação linear de \(V\) em \(Z\).

Por simplicidade, vamos definir um \textbf{operador linear} como sendo
uma transformação linear de um espaço vetorial \(V\) sobre ele mesmo.
Isto é, de \(V\) em \(V\).

Para os operadores \(U, T_1, T_2\) valem as seguintes propriedades:

\begin{enumerate}
\def\labelenumi{\arabic{enumi}.}
\item
  \(IU = UI = U\);
\item
  \(U(T_1 +T_2) = UT_1 + UT_2\) e \((T_1 + T_2)U = T_1U + T_2U\);
\item
  \(\alpha(UT_1)) = (\alpha U) T_1 = U(\alpha T_1)\), onde
  \(\alpha \in \mathbb{F}\).
\end{enumerate}

Uma transformação \(T: V \to W\) é dita \textbf{invertível} se existe
\(U : W \to V\) tal que \(UT\) é a identidade em \(V\) e \(TU\) é a
identidade em \(W\). Se \(T\) é invertível, então a função \(U\) é única
e denotada por \(T^{-1}\).

\(T\) e invertível se, e somente se,

\begin{itemize}
\item
  \(T\) é injetora. Ou seja \(T(v) = T(w) \implies v = w\).
\item
  \(T\) é sobrejetora. Isto é, \(Im(T) = W\).
\end{itemize}

Uma propriedade importante, análoga a matrizes, é que
\((UT)^{-1} = T^{-1}U^{-1}\).

Note que se pela linearidade de \(T\) vem que \(T(v-w) = T(v) - T(w)\).
Portanto, \(T(v) = T(w)\) se, e somente se, \(T(v-w) = 0\). Esse
resultado é bastante útil para verificar se uma transformação é
injetora.

Dizemos que \(T\) é \textbf{não singular} se \(T(v) = 0\) implica que
\(v = 0\). Ou seja, o núcleo de \(T\) é \(\{ 0 \}\). Assim, \(T\) é
injetora se, e somente se, \(T\) é não singular.

Uma extensão do fato acima é que a não singularidade de \(T\) é
equivalente a dizer que \(T\) leva cada subconjunto linearmente
independente de \(V\) em um subconjunto linearmente independente de
\(W\).

Caso \(dim(V) = dim(W)\), então o Teorema do Núcleo e Imagem nos garante
que as seguintes afirmações são equivalentes:

\begin{enumerate}
\def\labelenumi{\arabic{enumi}.}
\item
  \(T\) é invertível.
\item
  \(T\) é não singular (ou seja, é injetora).
\item
  \(T\) é sobrejetora.
\end{enumerate}

Assim, caso \(T\) seja um operador linear, é suficiente verificar que
\(T\) é injetora.

Observação: o conjunto dos operadores lineares em um espaço \(V\),
munido da operação de composição, é um \textbf{grupo}, conceito
importante da Álgebra. Além disso, o conjunto de vetores em um espaço
vetorial com a operação de soma de vetores é um \textbf{grupo
comutativo}.

\hypertarget{exercuxedcios-resolvidos-2}{%
\subsubsection{Exercícios Resolvidos}\label{exercuxedcios-resolvidos-2}}

\begin{enumerate}
\def\labelenumi{\arabic{enumi}.}
\item
  Encontre a inversa de \(T(x_1, x_2) = (x_1 + x_2, x_1)\), onde \(T\) é
  um operador linear em \(\mathbb{F}^2\).
\end{enumerate}

Primeiro verificamos que \(T\) é injetora.

Note que se \(T(x_1, x_2) = 0\), então temos

\[x_1 + x_2 = 0 \\
x_1 = 0\]

Ou seja, \(x_1 = x_2 = 0\). Portanto, \(Ker(T) = \{ 0 \}\).

Para verificar a sobrejetividade, considere
\((z_1, z_2) \in \mathbb{F}^2\). Queremos demonstrar que \((z_1, z_2)\),
que são vetores arbitrários, estão em \(Im(T)\). Ou seja, queremos
encontrar escalares \(x_1, x_2\) tais que

\[x_1 + x_2 = z_1 \\
x_1 = z_2\]

De onde temos \(x_1 = z_2\), \(x_2 = z_1 - z_2\).

Portanto, como \(T\) é bijetora, temos que \(T\) é invertível. O
processo para verificar a sobrejetividade nos dá a seguinte fórmula para
\(T^{-1}\):

\(T^{-1}(z_1,z_2) = (z_2, z_1-z_2)\)

Teorema do Núcleo e Imagem, Posto e Nulidade, Isomorfismo, Transformação
Inversa, Matriz de Representação

\hypertarget{produto-interno}{%
\subsection{Produto Interno}\label{produto-interno}}

Desigualdade de Cauchy-Schwarz, Norma, Ângulo e Ortogonalidade, Base
Ortogonal, Processo de Gram-Schmidt, Teorema Espectral

\hypertarget{autovalores-e-autovetores}{%
\subsection{Autovalores e Autovetores}\label{autovalores-e-autovetores}}

Multiplicidade Geométrica e Algébrica, Matrizes Especiais,
Diagonalização

\hypertarget{forma-canuxf4nica-de-jordan}{%
\subsection{Forma Canônica de
Jordan}\label{forma-canuxf4nica-de-jordan}}

\hypertarget{referuxeancias}{%
\subsection{Referências}\label{referuxeancias}}

HOFFMAN, K.; KUNZE, R. Linear Algebra. 2nd Edition.

\href{http://www.ime.unicamp.br/~pulino/ALESA/}{PULINO, P. Álgebra
Linear e suas Aplicações: Notas de Aula.}

\href{http://www.mat.ufmg.br/~regi/}{SANTOS, R. J. Álgebra Linear e
Aplicações.}

STRANG, G. Linear Algebra and its Applications.

COELHO, F. U.; LOURENÇO, M. L. Um curso de Álgebra Linear. 2. ed. São
Paulo: Edusp, 2005.

\href{https://www.youtube.com/playlist?list=PLZHQObOWTQDPD3MizzM2xVFitgF8hE_ab}{3Blue1Brown.
Essence of Linear Algebra.}

\href{/Volumes/Disco\%20Local/Google\%20Drive/Biblioteca\%20Computação,\%20Lógica\%20e\%20Matemática/Álgebra\%20Linear/Kenneth\%20M\%20Hoffman,\%20Ray\%20Kunze\%20-\%20Linear\%20Algebra\%20(2nd\%20Edition).pdf}{HOFFMAN,
K.; KUNZE, R. Linear Algebra. 2nd Edition.}

\href{/Volumes/Disco\%20Local/Google\%20Drive/Biblioteca\%20Computação,\%20Lógica\%20e\%20Matemática/Álgebra\%20Linear/ÁlgebraLinear\%20-\%20Pulino.pdf}{PULINO,
P. Álgebra Linear e suas Aplicações: Notas de Aula.}

\href{http://www.mat.ufmg.br/~regi/}{SANTOS, R. J. Álgebra Linear e
Aplicações.}

\href{/Volumes/Disco\%20Local/Google\%20Drive/Biblioteca\%20Computação,\%20Lógica\%20e\%20Matemática/Álgebra\%20Linear/Linear\%20Algebra\%20and\%20Its\%20Applications\%20(4ed)\%20by\%20Gilbert\%20Strang\%20(z-lib.org).pdf}{STRANG,
G. Linear Algebra and its Applications.}

COELHO, F. U.; LOURENÇO, M. L. Um curso de Álgebra Linear. 2. ed. São
Paulo: Edusp, 2005.

\href{https://www.youtube.com/playlist?list=PLZHQObOWTQDPD3MizzM2xVFitgF8hE_ab}{3Blue1Brown.
Essence of Linear Algebra.}

\end{document}
